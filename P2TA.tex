\documentclass[a4paper,10pt]{article}
\usepackage[utf8]{inputenc}

%opening
\title{Pay to Tagged Addresses (P2TA): Tagging blockchain transactions for queryability}
\author{Hans Robeers hrobeers@... twitter.com/hrobeers}

\begin{document}

\maketitle

\begin{abstract}
Multiple applications are using existing blockchains as a communication network.
Most of these implementations are scanning to blockchain for transactions matching their own format.
However, processing large blockchains to find application specific transactions can be expensive to execute.
This paper proposes a tagging mechanism, Pay to Tagged Addresses (P2TA), that allows quick lookup of application specific transactions based on the addition of deterministic output addresses.
\end{abstract}

\section{Introduction}
Blockchains are used increasingly for non-currency applications.
Examples of non-currency usage includes Colored Coins \_ref\_colored\_coins\_, PeerMessage \_ref\_peermessage\_, PSA \_ref\_psa\_ and multiple others.
These applications typically publish there own specific messages in the form of OP\_RETURN transactions on a third-party blockchain.
Scanning an entire blockchain for messages of a specific form becomes increasingly expensive while the blockchain grows.
Applications like PeerMessage \_ref\_peermessage\_ only rely on real-time transactions being relayed by the network and therefore do not suffer from blockchain growth.
However, applications like PSA \_ref\_psa\_ need know the entire history of transactions of a specific type to be able to validate asset ownership.
Therefore these type of applications greatly benefit from a system that allows fast querying of transactions holding a specific tag.

\end{document}
